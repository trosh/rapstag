\subsection*{Kriging Doe}

Dans cette partie notre but était de réaliser l’industrialisation
de module traitant des domaines de la statistique et de la
geostatistique. Il s'agit:
\begin{itemize}
\item Du Design of experiment(DoE)
\item Technique d’estimations du kriging et des ses dérivés(Kriging, CoKriging)
\end{itemize}
Nous nous sommes bas\'es sur des travaux réalisés par d’anciens
stagiaires de l’entreprise.

Notre mission consistait à :

\begin{itemize}
    \item Prendre en main leur projet en réalisant une lecture
        de leur rapport de stage, de la bibliographie associée
        et de rechercher des informations sur les thèmes.
    \item Restructurer leur codes sous un format plus industrialisable.
    \item Corriger les différentes erreurs qu’on aurait
        éventuellement rencontré et faire tourner leur implémentations.
    \item Établir des scripts de démonstration pour ces nouveaux modules.
    \item Et pour finir, comparer la nouvelle toolbox avec d’autres
        existantes (Dace, Scidoe) pour déterminer l’apport de ces
        travaux sur la bibliothèque de module Scilab.
\end{itemize}

Nous avons donc réalisé ces travaux pendant 1 mois et 2 semaine,
et présent\'e notre analyse sur le sujet à notre responsable
et au reste de l’équipe Scilab.

Pour l'industrialisation de ce code,
nous avons utilisé les connaissances relatives à
l’extensibilité du logiciel, acquises lors de la formation.
Nous avons donc utilisé un squelette de toolbox classique
de scilab ({\tt toolbox\_skeleton}) pour mettre en forme
le code de départ.

Après une étude approfondi et un ensemble de tests et comparaisons,
nous avons convenu que les modules déjà existant pour ces thématiques,
produisait de meilleures performances que ceux sur lesquels nous travaillions.

Néanmoins il était éventuellement intéressant d’ajouter aux modules existants,
certaines fonctionalités de ces travaux.

\subsubsection*{Exemple}
Pour la toolbox Dace traitant du kriging, nous avions ainsi l’idée d'intégrer
la technique du cokriging dite plus précise mais plus lourde en calcul. Aprés des
recherche sur le sujet, nous avons trouvé des implémentations de cette technique
sur le logiciel Matlab, et entammé des travaux de conversion.
Mais par souci de temps nous n'avons pas fini cette partie et nous nous sommes concentrer
sur le reste des objecitfs du stage.

