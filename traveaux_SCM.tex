\subsection*{Passage des ``complementary modules'' sur la version 6}

Ici notre but était de faire passer les modules de la listes SCM (Scilab complementary modules) dans la versions 6 de scilab. Il est nécessaire de passer par cette étape puisque pour la nouvelle “release”, le coeur du logiciel a été refondu, et des fonctions internes ont été modifiées ou supprimées. Ainsi plusieurs modules dépendants d’anciennes fonctions internes doivent être réécrit.

Nous avons commencé ces travaux en établissant une feuilles de calcul, ou nous avons répertorier l’ensemble des toolboxes à tester.

On peut diviser ces modules en 2 catégories:

Ceux composés uniquement de macros, c’est a dire des fichiers scripts Scilab d’extension .sce ou sci. Il s’agit de fonctions Scilab utilisant le langage du logiciel.

Ceux utilisant des gateways, qui sont des fichiers permettant d’appeler des codes d’autres langages de programmation, tels que le C, le C++ et le Java.

Au début de cette partie nous avons donc :

\begin{itemize}
\item Télechargé les modules
\item Evalué leurs pages d’aide et d’Overview
\item Tester leur fonctionnement sur Scilab 5
\item Essayer de les compiler sur Scilab 6
\end{itemize}

Pour le passage des toolboxes, nous avons donc commencé par les toolboxes composées uniquement de macros puisque leur passage était le plus facile à réaliser.

En général les tests sur une toolbox se fait en 3 étapes :

la compilation de la toolbox : Build

Le chargement : Load

Puis l'exécution des tests : test\_run

Pour chaque modules nous avons ainsi suivi les étapes et réalisé

les tests de fonctionnements. Même si le but est de passer tous

les tests avec succès, si autant de tests passent sur la version

6 que sur la 5, on peut considérer le module comme fonctionnel

(on vérifie toujours les erreurs et on ne les corrige pas

lorsqu’elles “ne sont pas de notre ressort” - tests de précision.

Au cours de ces tests nous avons rencontré différents problèmes. Nous avons pu régler certains grâce  à la  formation de départ et en recevant de l’aide des membres de l'équipe Scilab. Ils ont aussi été répertoriés au niveau de la fiche de portage établîtes.

Pour les modules contenant des gateways , la tache était plus difficile, car on peut rencontrer différents types de problèmes tels que :

Beaucoup de modules réalisent des appels à la stack qui était un espace mémoire alloué pour le logiciel, et qui n’existe plus dans la version 6. Il était donc nécessaire de changer ces appels, par de fonctions de L’API C par exemple.

Pour des raisons structurelles ou d’utilisation de la stack en fortran il est parfois nécessaire de réécrire toutes un ensemble de gateways dans le nouveau format C++

Certaines fonctions n’existent plus dans la version 6, donc le travail consiste à changer l’implémentation faisant appel à ces fonctions. 

Pour les gateways qui utilisent l’api\_scilab sans aucun appel à la stack, il suffit de changer le prototype des gateways (le controlleur de l’API n’est plus global mais local). Un include a change de nom pour éviter des conflits.

Dans le cas particulier du module sci-ipopt, la partie source de la toolbox et la partie gateway étaient liées. Des appels à la stack ainsi qu’a des fonction non existantes étaient réalisés. 

On a donc du appliquer toutes les étapes précédemment citées en plus de séparer les sources et les gateways.

