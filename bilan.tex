\section*{Bilan}

\subsection*{Bilan des travaux}
Durant ces travaux nous avons ainsi fait passer l’ensemble des toolboxes
de la liste des {\it SCM} \`a l’exception de celle dont on n'avait pas les droits,
et de {\tt sci-ipopt} sur lequel on travail encore. 
Ces nouvelles versions de modules sont présentement disponible
sur la forge de Scilab et Atoms, qui designe respectivement :

\begin{itemize}
\item un site ou la communauté de developpeur de modules peux les réccupérer,
grace à liens git et svn
\item une platorme de téléchargements directement branché à l'interface du logiciel.
\end{itemize}

\paragraph{Remarque}
Pour le moment il n'y a que les modules exclusivement composés
de {\it Macros} qui sont sur la plateforme Atoms.
Pour ceux utilisants les {\it gateways}, il est nécessaire de les
compiler sur plusieurs serveurs, que nous n'avons malheuresement
pas eu à disposition.
\\
Durant les phases de tests nous avons aussi rencontré des bugs liés au logiciel.On les
a reporté, sur le sites Bugzilla de Scilab, et des fois résolu, si c'était dans nos capacités.
Ces travaux nous ont donc permi de contribuer sur la distrubution et les extensions du produit.

\subsection*{Bilan personnel}
A la fin de notre stage, nous pouvons tirer un bilan de personnels positifs de notre experience chez
Scilab Enterprises. Même si son apport principale reste dans le developpement de logiciel, cette période
a été trés enrichissant au niveau gain de connaissance relatives à des problémes de nature physiques et mathématiques.
Elle nous aussi donner la possibilité d'apprendre à gérer des problématiques de gestion de projet informatique durant la
parti industrialisation de modules.
Ainsi ces mois nous ont permi d'accroitre considérablement nos connaissances dans le domaine de la Statistique
et de la Géostatisque, qui n'apartennait pas à notre domaine d'étude.

\paragraph{Remarque}
Avant de commencer cette partie, nous avons découper la mission en sous étapes d'un projet basique en
informatique. On établi un diagramme de Gantt, avec des possibilités de (go/no go) selon l'état du projet à
un moment donné. Nous avions aussi du présenter notre plan de projet à toute l'equipe Scilab avant de l'entamer. \newline
Pour la partie passage des modules, Même si pour leur test il n'est pas forcément nécessaire d'avoir
des connaissances accrues dans les domaines de ces extensions, pour la correction de certains
bug de une réflexion et études est des fois  necessaire.
Nous avons aussi amélioré notre maitrises du langage  C++ puisque le logiciel et la majorité des modules
sont programmés avec ce langages. Mais aussi du C, et du Fortan car dans certains cas de figure, on a du convertir ces codes
vers le C++ pour coller au nouveaux type de {\it gateways}.

