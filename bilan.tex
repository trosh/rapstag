\section*{Bilan}

\subsection*{Bilan des travaux}
Durant ces travaux nous avons ainsi fait passer l’ensemble des toolboxes
de la liste des {\it SCM} \`a l’exception de certaines moins prioritaires
et de {\tt sci-ipopt} sur lequel on travail encore.
Ces nouvelles versions de modules sont présentement disponibles
sur la forge de Scilab et Atoms, qui designent respectivement :

\begin{itemize}
    \item Une plateforme de r\'epositoires {\tt git}/{\tt svn} \`a l'usage
        grace à liens git et svn
    \item Une platorme de téléchargements directement branch\'ee à
        l'interface du logiciel.
\end{itemize}

\paragraph{Remarque}
Pour le moment il n'y a que les modules exclusivement composés
de {\it macros} qui sont sur la plateforme Atoms.
Pour ceux utilisant les {\it gateways}, il est nécessaire de les
compiler sur plusieurs serveurs, que nous n'avons malheuresement
pas eu à disposition.

Durant les phases de tests nous avons aussi rencontré des bugs liés
au logiciel.
On les a reportés sur le site Bugzilla de Scilab, et parfois résolus.
Ces travaux nous ont donc permis de contribuer sur la distribution
et les extensions du produit.

\subsection*{Bilan personnel}

\`A la fin de notre stage, nous pouvons tirer un bilan personnel positif
de notre experience chez {\sc Scilab Enterprises}.
Même si son apport principal reste dans le d\'eveloppement de logiciel,
cette période a été trés enrichissante au niveau gain de connaissances
relatives à des probl\`emes mathématiques.

Elle nous a aussi donn\'e la possibilité d'apprendre à gérer des
problématiques de gestion de projet informatique durant la
partie d'industrialisation de modules.
Ainsi ces mois nous ont permis d'accro\^itre considérablement nos
connaissances dans le domaine de la statistique et de la géostatisque,
qui n'appartenaient pas à notre domaine d'étude.

Avant de commencer cette partie, nous avons découp\'e la mission
en sous étapes d'un projet basique en informatique.
On a établi un diagramme de Gantt, avec des possibilités de
{\it go} / {\it no go} selon l'état du projet à des moments cl\'es.
Nous avions aussi d\^u présenter notre plan de projet à toute
l'equipe Scilab avant de l'entamer.

Lors du passage des modules, il n'est pas forcément nécessaire d'avoir
des connaissances accrues pour les tester.
N\'eanmoins, pour la correction de certains bugs,
une réflexion et une étude sont parfois n\'ecessaires.
Nous avons aussi amélioré notre ma\^itrise du C++
(Scilab et la majorité des modules sont programmés avec ce langage),
du C et du Fortan (lorsque les codes \'etaient en Fortran,
on a d\^u les convertir le C++ pour correspondre au nouveau
coeur de Scilab).
